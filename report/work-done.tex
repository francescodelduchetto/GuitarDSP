\chapter{Work Done}

\section{Application design}  %-----------------> Application design
In this stage of the work was designed the structure of the application,
studying the best way to organize the entities and the operations
between them.


The global operation is based on the \texttt{Model-View-Controller}
pattern that indicates the guidelines to write a good GUI-based
application. This pattern is very useful because it helps the programmer
to write highly reusable and extensible code when he has to work with
a user interface. Its philosophy is to split the general operation into
three parts: the management of the graphical interface (\texttt{View}), the
management of data and functionalities (\texttt{Model}) and the
communications between the \texttt{View} and the \texttt{Model}
(\texttt{Controller}).

\vspace{0.2in}

The application is composed by the following packages:
\begin{description}[noitemsep]
  \item[Main:] contains only the main class that launches the
    application.
  \item[View:] contains the classes that compose the user
    interface.
  \item[Model:] contains the classes that manage the internal
    operation, namely the streaming and processing of the sound.
  \item[Controller:] contains a class that translates user
    interactions with the \texttt{View} to commands for the \texttt{Model}, and also it
    modifies the user interface according to signals from the \texttt{Model}.
  \item[Effects:] contains the effects that implement the DSP
    algorithms to modify the input signal.
\end{description}

\newpage
Below there's a simple diagram that broadly shows how conceptually the
application works with the packages:

\tikzumlset{fill package=blue!10, font=\footnotesize}
\begin{center}
  \begin{tikzpicture}
    \begin{umlpackage}[x=8]{Main}
    \end{umlpackage}
    \begin{umlpackage}[x=4, y=-2]{Controller}
    \end{umlpackage}
    \begin{umlpackage}[y=-4]{View}
    \end{umlpackage}
    \begin{umlpackage}[x=8, y=-4]{Model}
    \end{umlpackage}
    \begin{umlpackage}[x=4, y=-6]{Effects}
    \end{umlpackage}
    \umluniassoc{Main}{Controller}
    \umluniassoc{Main}{Controller}
    \umluniassoc{View}{Controller}
    \umluniassoc{Controller}{View}
    \umluniassoc{Model}{Controller}
    \umluniassoc{Controller}{Model}
    \umluniassoc{Model}{Effects}
    \umluniassoc{Model}{Effects}
  \end{tikzpicture}
\end{center}

\texttt{Main} starts \texttt{Controller} which shows
the \texttt{View} and initializes the \texttt{Model}. User
interactions are delivered from \texttt{View} to
\texttt{Model} and vice versa by \texttt{Controller}. The
\texttt{Model} uses \texttt{Effects} to modify the input.

\newpage
\subsection{MVC pattern}  %-----------------> MVC pattern

As we said previously most of the entities of the application are
grouped into the three packages that make up the \texttt{MVC} pattern.\\
Below are shown these packages with the classes composing them.

 %-----------------> View
\vspace{0.2in}
%~ \noindent
\scalebox{0.9}{
\begin{tikzpicture}
  \begin{umlpackage}{View}
    \umlclass[x=4, y=0]{View}{}{
      \textbf{View(Controller)}\\
      \textbf{registerObserver(ViewObserver)}: void\\
      \textbf{getFileChooser()}: JFileChooser\\
      \textbf{getFileNameTxt()}: JTextField\\
      \textbf{getComponentIndex(Component)}: int\\
      \textbf{createEffectPanel(Effect)}: void\\
      \textbf{setComponentColor(Component, Color)}: void\\
      \textbf{setStartStopButtonText(String)}: void }
    \umlclass[x=-1, y=-4]{EffectPanel}{}{
      \textbf{EffectPanel(ViewObserver, Effect)} }
    \umlclass[x=6, y=-4]{WavFileFilter}{}{
      \textbf{getDescription()}: String\\
      \textbf{accept(File)}: boolean }
    \umlclass[type=interface, x=2, y=-9]{ViewObserver}{}{
      \textbf{getWindowListener()}: WindowListener\\
      \textbf{getOpenButtonListener()}: ActionListener\\
      \textbf{getAddButtonListener(JButton)}: ActionListener\\
      \textbf{getStartStopButtonListener()}: ActionListener\\
      \textbf{getInputParameterSldListener(}\\
      \ \ \ \ \textbf{InputParameter<Number>,JSlider,JLabel)}:\\
      \ \ \ \ ChangeListener\\
      \textbf{getUpperSwitchBtnListener(JPanel)}: ActionListener\\
      \textbf{getLowerSwitchBtnListener(JPanel)}: ActionListener\\
      \textbf{getRemoveEffectBtnListener(JPanel)}: ActionListener\\
      \textbf{getGraphBtnListener()}: ActionListener }
    \umlclass[x=-2, y=-14]{GraphView}{}{
      \textbf{GraphView()}\\
      \textbf{repaint(short[])}: void }
    \umlclass[x=5, y=-14]{GraphPanel}{}{
      \textbf{GraphPanel()}\\
      \textbf{paintComponent(Graphics)}: void }

    \umlunicompo{View}{WavFileFilter}
    \umlunicompo[anchor1=-110, anchor2=80]{View}{ViewObserver}
    \umluniassoc[anchor1=-105, anchor2=73]{View}{ViewObserver}
    \umlunicompo{GraphView}{GraphPanel}
  \end{umlpackage}
\end{tikzpicture} }

\newpage
\begin{description}
  \item[View:] extends \texttt{JFrame}. It shows the
    graphical interface on which the user can interact with the
    application.\\
    The \texttt{Controller}, to receive the user actions,
    must implement the \texttt{ViewObserver} interface and must register
    itself as an \texttt{Observer} of this class.
  \item[EffectPanel:] extends \texttt{JPanel}. It
    is the panel that is added when you click the button to add an
    effect.
    It allows the user to vary the parameters of the \texttt{Effect}
    thus customizing the output sound.
  \item[WavFileFilter:] extends \texttt{FileFilter}. It is used
    by the \texttt{JFileChooser} to filter out, in the window that
    allows user to select a file, the files that doesn't have the .wav
    extension.
  \item[ViewObserver:] it is an interface that must be
    implemented by the \texttt{Controller}. Therefore the
    \texttt{Controller} must provide the \texttt{Listeners} of the
    \texttt{View}'s components.
  \item[GraphView:] it shows the graph of the audio signal processed
    with the \texttt{Effects}. It contains only a \texttt{GraphPanel}
    that shows the signal.   
    It refreshes the graph every time that the \texttt{repaint(short[])}
    method is called, taking the new buffer that must be shown.
  \item[GraphPanel:] it's the \texttt{JPanel} that contains the graph
    of the signal. It draws the output signal according to the size of
    the \texttt{JFrame} which can be resized by the user.
    This function is made by overriding the \texttt{paintComponent(Graphics)}
    method.
\end{description}

\newpage
%-----------------> Model
\noindent
%~ \tikzumlset{font=\scriptsize}
\begin{tikzpicture}
  \begin{umlpackage}{Model}
    \umlclass[x=0, y=1]{Model}{ }{
      \textbf{Model(Controller)}\\
      \textbf{getAvailableEffects()}: List<Class\\
      \ \ \ \ <? extends Effect> >\\
      \textbf{getEffects()}: List<Effect>\\
      \textbf{addEffect(Effects)}: void\\
      \textbf{removeEffect(int)}: void\\
      \textbf{exchangeEffects(int, int)}: void\\
      \textbf{getInputAttenuation()}:\\
      \ \ \ \ InputParameter<Double>\\
      \textbf{startStream(String)}: void\\
      \textbf{stopStream()}: void }
    \umlclass[x=3, y=-12]{Streamer}{}{
      \textbf{Streamer(SourceDataLine, File,AudioFormat,}\\
      \ \ \ \ \textbf{InputParameter<Double>, Controller, Model)}\\
      \textbf{run()}: void\\
      \textbf{stopStream()}: void }
    \umlclass[x=7, y=0]{AudioSettings}{}{
      \textbf{getAudioSettings()}: AudioSettings\\
      \textbf{getAudioFormat()}: AudioFormat\\
      \textbf{getBufferFrames()}: int\\
      \textbf{getBufferLength()}: int\\
      \textbf{getSampleRate()}: int\\
      \textbf{getSampleSizeInBits()}: int\\
      \textbf{getShortBufferLength()}: int\\
      \textbf{getNumChannels()}: int\\
      \textbf{getSigned()}: boolean\\
      \textbf{getBigEndian()}: boolean\\ }
    \umlclass[x=3.5, y=-6.5]{InputParameter<N>}{}{
      \textbf{InputParameter(String, N, N, N)}\\
      \textbf{getValue()}: N\\
      \textbf{getMaxValue()}: N\\
      \textbf{getMinValue()}: N\\
      \textbf{getIntValue()}: int\\
      \textbf{getIntMaxValue()}: int\\
      \textbf{getIntMinValue()}: int\\
      \textbf{getName()}: String\\
      \textbf{setValue(N)}: void\\
      \textbf{setIntValue(int)}: void }
    \umlassoc[anchor1=110, anchor2=155]{Model}{Streamer}
    \umluniassoc[geometry=|-,anchor1=160, anchor2=190]{Streamer}{InputParameter<N>}
    \umluniassoc[geometry=|-, anchor1=240, anchor2=170]{Model}{InputParameter<N>}
    \umluniassoc{Model}{AudioSettings}
    \umluniassoc[anchor1=20]{Streamer}{AudioSettings}
    \umluniassoc[angle1=70, angle2=110]{AudioSettings}{AudioSettings}
  \end{umlpackage}
\end{tikzpicture}

\newpage
\begin{description}
  \item[Model:] it manages the whole operation of the application.
    It is able to start a new \texttt{Streamer} and to stop it, also it
    manages the \texttt{Effects} allowing the user to add, remove and exchange them. 
  \item[Streamer:] extends \texttt{Thread}. When started
    it creates a new thread which flows the audio from the input stream to
    the output line applying the effects added by the user. Every time
    it reads a new buffer from the input line it asks to the \texttt{Model}
    the list of \texttt{Effects} that it must apply, allowing the user
    to add, remove and exchange effects during the streaming of the file.
    The \texttt{Streamer} also call the \texttt{updateGraph(short[])}
    method of \texttt{Controller} every time that a new buffer is
    processed, refreshing the graph.
  \item[InputParameter<N>:] extends \texttt{Observable}. It %TODO
    encapsulates the idea of parameter that can be varied at run-time 
    modifying the state of an \texttt{Effect}. For example it is used
    by the \texttt{OverdriveEffect} to allow the user to modify the
    power of distortion.\\
    It is a generic class where \texttt{N} must extends \texttt{Number}.
    It provides however methods to access the value as \texttt{int}
    number because the \texttt{JSlider}, which is used to vary its
    value, works with \texttt{int} numbers.\\
    It also notifies its \texttt{Observers} when its value is changed. 
    An \texttt{Effect} can register itself as \texttt{Observer} of their
    \texttt{InputParameters}.
  \item[AudioSettings:] it follows the agreement of the
    \texttt{Singleton} pattern, in fact its
    constructor is private and the only existing instance can be taken
    only by calling the \texttt{AudioSettings.getAudioSettings} method.\\
    I chose to make it a \texttt{Singleton} pattern because there are no reasons
    to have more than one instance of this class at the same time and
    also to avoid having to pass its reference to every object that
    needs it.\\
    It provides methods to retrieve the audio technical information, such
    as the frame size or the buffer length used.    
\end{description}


\newpage
%-----------------> Controller
%~ \noindent
\begin{tikzpicture}
  \begin{umlpackage}{Controller}
    \umlclass[x=0, y=0]{Controller}{}{
      \textbf{Controller()}\\
      \textbf{showErrorDialog(String)}: void\\
      \textbf{streamStarted()}: void \\
      \textbf{streamStopped()}: void \\
      \textbf{createGraphView()}: void \\
      \textbf{updateGraph(short[])}: void \\
      \textbf{getInputAttenuation()}: InputParameter<Double> \\
      \textbf{getWindowListener()}: WindowListener \\
      \textbf{getOpenButtonListener()}: ActionListener \\
      \textbf{getAddButtonListener(JButton)}: ActionListener \\
      \textbf{getStartStopButtonListener()}: ActionListener \\
      \textbf{getPopupMenuItemListener(Class<Effect>)}: PopupMenuItemListener \\
      \textbf{getInputParameterSldListener(InputParameter<Number>,}\\
      \ \ \ \ \textbf{JSlider,JLabel)}: ChangeListener \\
      \textbf{getUpperSwitchBtnListener(JPanel)}: ActionListener \\
      \textbf{getLowerSwitchBtnListener(JPanel)}: ActionListener \\
      \textbf{getRemoveEffectBtnListener(JPanel)}: ActionListener \\
      \textbf{getGraphButtonListener()}: ActionListener }
    \umlclass[x=0, y=-7]{PopupMenuItemListener}{}{
      \textbf{PopupMenuItemListener(Class<Effect>)}\\
      \textbf{actionPerformed(ActionEvent)}: void }
    \umlunicompo{Controller}{PopupMenuItemListener}
  \end{umlpackage}
\end{tikzpicture}

\begin{description}
  \item[Controller:] its constructor creates a new \texttt{View}
    and a new \texttt{Model} acting as a link between them. It
    implements \texttt{ViewObserver} and therefore provides the
    \texttt{Listeners} for the \texttt{View}.
    It also supplies methods to allow the \texttt{Model} to report
    when a stream is started or stopped and methods to create and to update
    the \texttt{GraphView}.
  \item[PopupMenuItemListener:] it handles the operations to do
    when an item, contained in the pop-up that allows the user to
    select an effect, is pressed. It says the \texttt{Model} to create
    a new \texttt{Effect} and it says the View to add a new
    \texttt{EffectPanel} for the \texttt{Effect} created.
\end{description}

%-----------------> Effects
\subsection{Effects}
%~ \noindent
\scalebox{0.9}{
\begin{tikzpicture}
  \begin{umlpackage}{Effects}
    \umlclass[type=interface, x=0, y=-6]{Effect}{}{
      \textbf{process(short[], int)}: void\\
      \textbf{initialize()}: void\\
      \textbf{getInputParameters()}: List\\
      \ \ \ \ <InputParameter\\
      \ \ \ \ <? extends Number>> }
    \umlclass[type=annotation, x=0, y=-1.5]{Attributes}{}{
      \textbf{name()}: String\\
      \textbf{isShowable()}: boolean }
    \umlclass[x=8, y=-11]{ButterworthLPFilterEffect}{}{
      \textbf{ButterworthLPFilterEffect()}\\
      \textbf{process(short[],int)}:void\\
      \textbf{getInputParameters()}:List<\\
      \ \ \ \ InputParameters<\\
      \ \ \ \ ? extends Numbers>>\\
      \textbf{initialize()}:void\\
      \textbf{update(Observable,Object)}: void }
    \umlclass[x=0, y=-11]{OverdriveEffect}{}{
      \textbf{OverdriveEffect()}\\
      \textbf{process(short[],int)}:void\\
      \textbf{getInputParameters()}:List<\\
      \ \ \ \ InputParameters<\\
      \ \ \ \ ? extends Numbers>>\\
      \textbf{initialize()}:void\\
      \textbf{update(Observable,Object)}: void }
    \umlclass[x=8, y=-6.5]{ReverbEffect}{}{
      \textbf{ReverbEffect()}\\
      \textbf{process(short[],int)}:void\\
      \textbf{getInputParameters()}:List<\\
      \ \ \ \ InputParameters<\\
      \ \ \ \ ? extends Numbers>>\\
      \textbf{initialize()}:void\\
      \textbf{update(Observable,Object)}:void }
    \umlclass[x=7, y=-1.8]{FlangerEffect}{}{
      \textbf{FlangerEffect()}\\
      \textbf{process(short[],int)}:void\\
      \textbf{getInputParameters()}:List<\\
      \ \ \ \ InputParameters<\\
      \ \ \ \ ? extends Numbers>>\\
      \textbf{initialize()}:void\\
      \textbf{update(Observable,Object)}:void }
    \umlclass[x=0, y=3]{DelayEffect}{}{
      \textbf{DelayEffect()}\\
      \textbf{process(short[],int)}:void\\
      \textbf{getInputParameters()}:List<\\
      \ \ \ \ InputParameters<? extends Numbers>>\\
      \textbf{initialize()}:void\\
      \textbf{update(Observable,Object)}: void }
    \umlclass[x=8, y=3]{DelayLine}{
      \texttt{MAX\_DELAY\_LENGTH}:int }{
      \textbf{DelayLine(}\\
      \ \ \ \ \textbf{InputParameter<Integer>)}\\
      \textbf{getResponse(short[],int)}:short[]\\
      \textbf{getDelayLength()}:\\
      \ \ \ \ InputParameters<Integer>\\
      \textbf{getFeedback()}:\\
      \ \ \ \ InputParameter<Double>\\
      \textbf{emptyDelayBuffer()}:void }
    \umlimpl{ButterworthLPFilterEffect}{Effect}
    \umlimpl{FlangerEffect}{Effect}
    \umlimpl[anchor1=-90, anchor2=70]{DelayEffect}{Effect}
    \umlimpl{ReverbEffect}{Effect}
    \umlimpl{OverdriveEffect}{Effect}
    \umluniassoc{DelayEffect}{DelayLine}
    \umluniassoc[geometry=-|-, arm1=-1cm, anchor1=-170, anchor2=170]{ReverbEffect}{ButterworthLPFilterEffect}
    \umluniassoc{OverdriveEffect}{ButterworthLPFilterEffect}
    \umluniassoc[anchor1=40, anchor2=-45]{ReverbEffect}{DelayLine}
    \umlunicompo{Effect}{Attributes}
    \end{umlpackage}
\end{tikzpicture}
}

\newpage

\begin{description}
  \item[Effect:] it is an \texttt{Interface} that extends
    \texttt{Observer} because it can register itself to receive a
    notification by some of its \texttt{InputParameter}.\\
    Each class that implement this interface must be able to:
    \begin{itemize}
      \item process a given buffer of shorts;
      \item initialize itself resetting its environment;
      \item return the \texttt{InputParameters} that the user can modify.
    \end{itemize}
  \item[Attributes:] it is an \texttt{Annotation} that stores
    the name of the \texttt{Effect} and also it allows to know if it's
    show-able to the user since some effects are for inner use and the
    user can't use it directly.
    This annotations is necessary because allows,
    through \texttt{reflection}, to know which effects the user can
    select before they are instantiated.\\
    This \texttt{Annotation} allow, in the future, to add new
    \texttt{Effects} without modifying the \texttt{View}.
  \item[DelayLine:] it isn't an effect because it doesn't modifies
    the original sound but it simply stores, in a circular buffer, an
    infinite series of the given sounds according to these parameters:
    \begin{description}
      \item[Delay length:] it defines the length of the circular
        buffer, changing the perception of the sound reflection's
        distance.
      \item[Feedback:] it defines the amount of the original
        sound that must be stored. Its value can vary in the range $[0,1]$, 
        if the value is $0$ the effect doesn't store any delay, if the value
        is $1$ it stores the sound with the same volume as the original.
    \end{description}
    To get the dealy response associated to the original buffer
    you must call \texttt{getResponse(originalBuffer, bufferLength)}.
  \item[DelayEffect:] it is a simple effect that use a
    \texttt{DelayLine} to perform an echo effect on the original sound.
    It allows the user to modify the length of the delay and the gain value
    of the echoes. \cite{comb}.
  \item[FlangerEffect:] is an \texttt{Effect} that mixes the
    original sound with a delayed copy of the original signal. The delay
    time is continuously varied by a low frequency.\\
    This effect doesn't use any delay lines because the \texttt{DelayLine}
    keeps an infinite series of previously echoes while it needs only
    the previous delayed sound.\cite{flanger}
    The parameters that the user can modify are:
    \begin{description}
      \item[Frequency:] the frequency of the $\sin$ function that
      continuously modify the delay length.
      \item[Excursion:] the range in which the delay can vary.
      \item[Depth:] modifies the percentage of delayed sound
      over the original sound.
    \end{description}
  \item[ButterworthLPFilterEffect:] it's a low pass filter that
    filters the frequencies above a given bound which can be varied by
    the user\cite{butterworth}.
    It isn't showable to the user, it's only used by the
    \texttt{OverdriveEffect} and by the \texttt{ReverbEffect}.
  \item[OverdriveEffect:] it applies a non-linear distortion
    function to the original sound\cite{overdrive}. The ``sound'' of the
    distortion is varied through two parameters:
    \begin{description}
      \item[Drive:] modifies the power of the distortion, if its
        value is equal to $0$ the output sound is the same as the
        original signal and higher is the value more the sound is
        amplified and like a square wave.
      \item[Cutoff:] manages the \texttt{ButterworthLPFilterEffect}
        that cut the higher frequencies that in a real situation doesn't
        occur and allows user to change the ``color'' of the distortion.
    \end{description}
    Using the \texttt{GraphView} we can clearly see the effect of these
    two parameters on the original signal.
  \item[ReverbEffect:] it simulates the echoes that occurs
    for example in a church or in a concert hall. Often this effect is
    added to a registration to make the sound less dry and more pleasant.\\
    The algorithm implemented is based on the
    \textit{Schroeder Reverberator}\cite{reverb} that suggest to apply
    two stages of delay to the incoming sound:
    \begin{description}
      \item[Early reflections:] is implemented using some
        delay lines chained in series. The delay lengths are very small
        to increase the density of the audio impulses and different
        from each other to make a good spatialization of the echoes.
      \item[Late reflections:] is implemented using some delay
        lines chained in parallel. Their delay lengths are longer than
        the previous and varying them we can simulate the size of the
        rooms in which the reverb occurs.\\
        The late reflections are filtered with a
        \texttt{ButterworthLPFilterEffect} to simulate the air
        absorption of the high frequencies.
    \end{description}
  
\end{description}
