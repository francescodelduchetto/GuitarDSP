\chapter{Considerations}

\section{Workflow}
\begin{enumerate}
  \item The design of the project started with a study of the Java
  Sound API to understand how it works and how its tools could meet my
  needs.
  \item Then I designed an initial architecture of the entities starting
  to write some code.
  \item At this point I had to search some resources on the web that
  helped me to understand how I actually had to implement the effects.
\end{enumerate}

\subsection{In-development changes}
During the development of the project there was few changes from the
initial intentions:
\begin{enumerate}
  \item I implemented the effects extending the abstract class
    \texttt{Control} provided by the \texttt{javax.sound.sampled}
    package which represents a control that the line requested can have but
    then I choose to refactor the effects under the new interface
    \texttt{Effect} to have more movement's freedom and
    because conceptually, in my opinion, it is better a choice.
  \item I had decided to make the \texttt{View} and the
    \texttt{Model} unique using the \texttt{Singleton} pattern
    but then I changed this setting to allow in the future to add some
    of these if necessary.
  \item I decided to remove the audio settings in the
    \texttt{Model} refactoring them in the entity
    \texttt{AudioSettings} in which I will be able to add, in the future,
    the methods to modify these settings according to user's needs.
  \item Lately I added the functionality that allows the user to see the
    output audio signal in a graph.
\end{enumerate}

\section{Possible troubles}

It is recommended not to use an overdrive effect before a delay
or a reverb effect because the overdrive saturation, that is replicated
many time by the delay, will overflow the buffers producing only noise.
